Punkty $A$ i $B$ należą do obwodu podstawy stożka o wierzchołku $S$. Wysokość $SO$ stożka ma długość $4$, kąt $\angle AOB$ jest prosty i $AB=6$. Obwód trójkąta $ABS$ jest równy:
Kulę o promieniu $7$ przecięto płaszczyzną i w przekroju otrzymano koło o promieniu $2$. Odległość środka kuli od tego przekroju wynosi:
Powierzchnia boczna walca po rozwinięciu jest prostokątem, którego przekątna o długości $8$ tworzy z bokiem równym wysokości walca kąt o mierze $60^{\circ}$. Oblicz objętość walca.
Ołowiany walec o promieniu $12$ cm i wysokości $5$ cm przetopiono na kule o promieniu $3$ cm. Ile kul otrzymano?
Przekrój osiowy stożka jest trójkątem równobocznym o polu równym $18$. Oblicz pole powierzchni bocznej tego stożka.
Pole powierzchni kuli równa się $676\pi$. Dwa wzajemnie równoległe przekroje kuli, leżące po tej samej stronie koła wielkiego do nich równoległego, mają pola $25\pi$ i $144\pi$. Oblicz odległość między tymi przekrojami
Stożek przeciąto płaszczyznami równoległymi do podstawy i dzielącymi jego wysokość na $4$ równe części. Wykaż, że obwody powstałych przekrojów tworzą ciąg arytmetyczny.
Walec ma promień podstawy równy $r$. W podstawie dolnej tego walca poprowadzono cięciwę $AB$ odległą od środka podstawy o $\frac{r}{2}$. Prostokąt $ABCD$ jest przekrojem walca płaszczyzną równoległą do jego osi i $\frac{AB}{BC}=\frac{1}{2}$ oraz $AC=4\sqrt{15}$. Oblicz objętość tego walca.
Rozwinięcie powierzchni bocznej stożka jest wycinkiem kołowym o kącie środkowym $\alpha$. Kąt ten oparty jest na łuku o długości $a$. Oblicz objętość tego stożka.
Punkty $A$ i $B$ należą do obwodu podstawy stożka o wierzchołku $S$. Wysokość $SO$ stożka ma długość $4$, kąt $\angle AOB$ jest prosty i $AB=6$. Obwód trójkąta $ABS$ jest równy:
Kulę o promieniu $7$ przecięto płaszczyzną i w przekroju otrzymano koło o promieniu $3$. Odległość środka kuli od tego przekroju wynosi:
Powierzchnia boczna walca po rozwinięciu jest prostokątem, którego przekątna o długości $8$ tworzy z bokiem równym wysokości walca kąt o mierze $60^{\circ}$. Oblicz objętość walca.
Ołowiany walec o promieniu $12$ cm i wysokości $6$ cm przetopiono na kule o promieniu $3$ cm. Ile kul otrzymano?
Przekrój osiowy stożka jest trójkątem równobocznym o polu równym $18$. Oblicz pole powierzchni bocznej tego stożka.
Pole powierzchni kuli równa się $676\pi$. Dwa wzajemnie równoległe przekroje kuli, leżące po tej samej stronie koła wielkiego do nich równoległego, mają pola $25\pi$ i $144\pi$. Oblicz odległość między tymi przekrojami
Stożek przeciąto płaszczyznami równoległymi do podstawy i dzielącymi jego wysokość na $4$ równe części. Wykaż, że obwody powstałych przekrojów tworzą ciąg arytmetyczny.
Walec ma promień podstawy równy $r$. W podstawie dolnej tego walca poprowadzono cięciwę $AB$ odległą od środka podstawy o $\frac{r}{2}$. Prostokąt $ABCD$ jest przekrojem walca płaszczyzną równoległą do jego osi i $\frac{AB}{BC}=\frac{1}{2}$ oraz $AC=4\sqrt{15}$. Oblicz objętość tego walca.
Rozwinięcie powierzchni bocznej stożka jest wycinkiem kołowym o kącie środkowym $\alpha$. Kąt ten oparty jest na łuku o długości $a\pi$. Oblicz objętość tego stożka.
Liczby rzeczywiste $a,b,c,d$ spełniają równości $$(a+b)(c+d)=(a+c)(b+d)=(a+d)(b+c).$$ Udowodnij, że co najmniej trzy z liczb $a,b,c,d$ są równe.
Dany jest trójkąt $ABC$, w którym $AC=BC$. Punkt $D$ leży na boku $AC$, a punkt $E$ wybrany jest w taki sposób, że czworokąt $ABED$ jest równoległobokiem. Symetralna odcinka $DE$ przecina odcinek $BC$ w punkcie $P$. Udowodnij, że $AD=CP$.  
Rozstrzygnij, czy można tak dobrać grupę $2017$ osób, by każda z nich miała w tej grupie dokładnie tyle samo znajomych co nieznajomych. Przyjmujemy, że relacja znajomości jest symetryczna, tzn. jeśli osoba $A$ zna osobę $B$, to również osoba $B$ zna osobę $A$.
Znajdź wszystkie trójki liczb pierwszych $(p,q,r)$, dla których $$\frac{p^2+q}{q^2+p}=r\textrm{ oraz }\frac{p^2+r}{r^2+p}=q.$$
W trójkącie ostrokątnym $ABC$ symetralna boku $AC$ i wysokość poprowadzona do boku $BC$ przecinają się na dwusiecznej kąta $ACB$. Wykaż, że symetralna boku $BC$ i wysokość poprowadzona do boku $AC$ również przecinają się na dwusiecznej kąta $ACB$.
Czy istnieją takie cztery dodatnie liczby całkowite, że dowolne dwie z nich mają największy wspólny dzielnik większy od $1$, a dowolne trzy z nich mają największy wspólny dzielnik równy $1$?
Punkt $M$ jest środkiem boku $AB$ trójkąta $ABC$. Na odcinku $CM$ znajduje się taki punkt $D$, że $AC=BD$. Wykaż, że $$\angle MCA=\angle MDB.$$ 
Każde pole tablicy o wymiarach $8\times 8$ pomalowano na biało lub czarno. Okazało się, że w każdym kwadracie o wymiarach $3\times 3$ złożonym z całych pól tej tablicy, znajduje się parzysta liczba czarnych pól. Jaka jest najmniejsza możliwa liczba białych pól w całej tablicy?
Punkty $E$ i $F$ leżą odpowiednio na bokach $BC$ i $CD$ prostokąta $ABCD$, przy czym trójkąt $AEF$ jest równoboczny. Wykaż, że suma pól trojkątów $ABE$ i $ADF$ jest równa polu trójkąta $CEF$.
Udowodnij, że istnieje nieskończenie wiele par liczb naturalnych $(a,b)$, dla których $$NWD(a^2+1, b^2+1)=a+b.$$
Jakim procentem liczby $12$ jest $40\%$ liczby $20$?
Ile liczb całkowitych spełnia nierówność $|x-3|\leq 5$?
Wykaż, że dla każdej pary liczb rzeczywistych $a,b$ prawdziwa jest nierówność $$6x^2+y^2 +4x-6y+10\geq 0?$$
Dane są zbiory $A=\{x\in\mathbb{R}: 2<\frac{x}{2}+3,5\leq 6\}$ oraz $B=\{x\in\mathbb{R}:|x+3|\leq 3\}$. Wyznacz zbiór $A\cap B$.
Funkcja $f$ przyporządkowuje każdej liczbie naturalnej dwukrotność jej największego dzielnika pierwszego. Wyznacz wartość różnicy $f(1001)-f(210)$. 
Rozwiąż nierówność: $$||2x-9|-10|\leq 7.$$
Funkcja $f$ jest określona na zbiorze liczb naturalnych trzycyfrowych i przyporządkowuje każdej takiej liczbie iloczyn jej cyfr. Dla ilu argumentów funkcja $f$ przyjmuje wartość $28$?
Funkcja $f(x)=ax^2+8x+c$ dla argumentu $1$ przyjmuje wartość o $6$ większą niż dla argumentu $4$. Do wykresu funkcji $f$ należy początek układu współrzędnych. Wyznacz zbiór wartości funkcji $f$.
Dla jakich wartości parametru $m$ równanie $$x^2+(m-5)x+m^2+m+\frac{1}{4}=0$$ ma dwa pierwiastki, oraz stosunek sumy tych pierwiastków do ich iloczynu wynosi $2$.
Jeżeli prosta będąca wykresem funkcji liniowej $f$ przecina oś $OX$ w punkcie o odciętej $x=-2$ oraz przechodzi przez punkt $(0,-6)$, to:
Liczba rozwiązań równania $\frac{5-x}{x^2-25}=0$ jest równa:
Niech $A$ będzie zbiorem rozwiązań nierówności $x-\frac{x-4}{3}<2x-8$, a $B$ zbiorem rozwiązań nierówności $x(x-4)+1\geq (x+1)^2-6x$. Wyznacz $A\cap B$.
Określ, czy prosta opisana równaniem $3x-7y=2$ jest równoległa do prostej przechodzącej przez punkty $A(-1,2)$ i $B(6,5)$.
Rozwiąż równanie $\frac{x}{4} (\frac{x^2}{4}-9)=0$. Czy któreś rozwiązanie należy do zbioru rozwiązań nierówności $(-4-x)^2\leq 9$?
Funkcja kwadratowa $f(x)=x^2+6x+c$ ma dokładnie jedno miejsce zerowe. Wyznacz $c$ i wyznacz najmniejszą i największą wartość funkcji $f$ w przedziale $[-4,0]$.
Zbiorem wszystkich argumentów, dla których funkcja $f(x)=\frac{4-2x}{3x+d}$ przyjmuje wartości ujemne jest zbiór $(-\infty , -1)\cup (2,\infty)$. Wyznacz $d$.
Wyznacz te wartości $x\in [0;2\pi]$, dla których liczby $\frac{1}{2}$, $\sin x$, $\sin 2x$ są kolejnymi wyrazami ciągu geometrycznego.
Niech $a_n$, dla $n\geq 1$, będzie resztą z dzielenia wielomianu $w_n(x)=\left(2x^2-3x-\frac{11}{2}\right)^n$ przez dwumian $x+1$. Oblicz sumę ośmiu początkowych wyrazów ciągu $(a_n)$.
Rozwiąż nierówność $f(x-1)-f(x+1)>6$, gdzie $f(x)=4-\frac{3}{x}$.
Wyznacz zbiór wartości funkcji $f(x)=3-4\sin x-4\cos^2 x$.
Naszkicuj wykres funkcji $f(x)=\left| \frac{x+2}{x-2}\right|$, a następnie określ liczbę rozwiązań równania $f(x)=p$ w zależności od wartości parametru $p$.
Naszkicuj wykres funkcji $f(x)=\frac{|x-3|}{x+1}$, a następnie określ liczbę rozwiązań równania $f(x)=p$ w zależności od wartości parametru $p$.
Wiedząc, że jednym z miejsc zerowych funkcji $f(x)=\frac{x^3+bx^2-13x-10}{x+1}$ jest $5$, rozwiąż nierówność $f(x)\geq 0$ .
Łódź musi popłynąć $60$ km w dół rzeki, a następnie $10$ km w górę rzeki. Prędkość prądu rzeki wynosi $5$ km/godz. Jaka powinna być prędkość łodzi, aby cała podróż nie trwała dłużej niż $10$ godzin?
Łódź musi popłynąć $24$ km w dół rzeki, a następnie $48$ km w górę rzeki. Prędkość prądu rzeki wynosi $5$ km/godz. Jaka powinna być prędkość łodzi, aby cała podróż nie trwała dłużej niż $6$ godzin?
Rozwiąż równanie $$ \cos 2x + \sin 2x +1 =0.$$
Rozwiąż równanie $$ \sin x\cdot\tg x -\sqrt{3}=\tg x -\sqrt{3}\sin x.$$
Rozwiąż równanie $$ \tg x + \ctg x =4\sin 2x.$$
Rozwiąż równanie $$ \ctg x + \frac{\sin x}{1+\cos x} =2.$$
Jakim procentem liczby $12$ jest $40\%$ liczby $20$?
Ile liczb całkowitych spełnia nierówność $|x-3|\leq 5$?
Wykaż, że dla każdej pary liczb rzeczywistych $a,b$ prawdziwa jest nierówność $$6x^2+y^2 +4x-6y+10\geq 0$$.
Dane są zbiory $A=\{x\in\mathbb{R}: 2<\frac{x}{2}+3,5\leq 6\}$ oraz $B=\{x\in\mathbb{R}:|x+3|\leq 3\}$. Wyznacz zbiór $A\cap B$.
Funkcja $f$ jest określona na zbiorze liczb naturalnych trzycyfrowych i przyporządkowuje każdej takiej liczbie iloczyn jej cyfr. Dla ilu argumentów funkcja $f$ przyjmuje wartość $15$? 
